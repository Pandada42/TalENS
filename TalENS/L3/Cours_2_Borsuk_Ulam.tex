\documentclass{beamercours}

\title{Cours TalENS 2023-2024}
\subtitle{Goûters, Impôts et Chaleur} 
\date{16 Décembre 2023}


\begin{document}
\maketitle

\section{Mise en Situation}
\subsection{Le Problème}
\begin{frame}
    \frametitle{Vie en Communauté}
    Placez vous dans la situation suivante : \\
    \begin{center}
        \visible<2-> Avec plusieurs de vos amis, vous avez décidé de faire un gâteau. \\
        \visible<3-> Une fois celui-ci cuit, vient le moment de le découper.\\
        \visible<4-> Cependant, vous n'avez pas tous aussi faim les uns que les autres. \\
        \visible<5-> Comment faire pour le découper sans que personne ne soit lésé et que le découpeur ne soit assailli pour ses préférences dans le groupe d'amis.
    \end{center}
\end{frame}

\subsection{Mathématiquement}
\begin{frame}
    \frametitle{Modélisation}
    \visible<1-> On modélise l'ensemble des mangeurs de gâteau par le segment d'entier $\onen{n}$\\
    \visible<2-> On note $\mathfrak{S}_{n}$ l'ensemble des permutations de ce segment d'entier. \\
    \visible<3-> On représente la faim du mangeur $i$ par une mesure de probabilité $\mu_{i}$ de densité $f_{i}$. \\
    \visible<4-> On cherche une partition $X_{1}, \ldots, X_{n}$ du gâteau, que l'on modélise par l'ensemble $\left[0, 1\right]$.
\end{frame}

\begin{frame}
    \frametitle{Théorème de Partage de Gâteau}
    \begin{theorem}[De Partage de Gâteau]
        Pour toutes fonctions de densités $f_{i}$ et permutations $\sigma \in \mathfrak{S}_{n}$, on considère le système $(\star)$ suivant : 
        \begin{equation*}\tag{$\star$}
            \int_{0}^{x_{1}}f_{\sigma(1)}(x) \mathrm{d}x = \int_{x_{1}}^{x_{2}}f_{\sigma(2)}(x)\mathrm{d}x = \ldots = \int_{x_{n}}^{1}f_{\sigma(n)}(x) \mathrm{d}x 
        \end{equation*}  
        Celui-ci a une solution. 
    \end{theorem}
\end{frame}


\end{document}


\documentclass{beamercours}

\author{Matthieu Boyer}
\title{Cours TalENS 2023-2024}
\subtitle{Kilométrage, Hébreu, Monstruosité et Cartes à Jouer}
\date{25 Novembre 2023}

\begin{document}
\maketitle
\section*{Introduction}
\begin{frame}
    \frametitle{Comptons un peu}
    \begin{itemize}
        \item \visible<1->{Piétonnement : $0, 1, 2, 3, 4, 5, \ldots$}
        \item \visible<2->{En courant alors ? $1, 10, 100, 1000, 10000, 100000, \ldots$}
        \item \visible<3->{Du nerf ! $1, 1, 2, 6, 24, 120, 720, 5040, 40320, 362880, 3628800, \ldots$}
    \end{itemize}
    \visible<4->{Bon, on est quand même loin du bout non ?}
\end{frame}

\begin{frame}
    \frametitle{Intuitons}
    \visible<1->{L'infini, c'est long, surtout vers le bord...}\\

    \visible<2->{Mais bon, est-ce que l'infini, c'est si bien défini ? Est-ce que tout les infinis sont les mêmes ?} \\

    \visible<3->{Il y a une infinité de nombres pairs (et de nombres impairs), mais, puisque tout nombre pair est un nombre entier, il devrait y en avoir moins.}
\end{frame}


\section{Cantor arrive en ville}
\subsection{Un peu de théorie des ensembles... Mais vraiment un peu...}

\begin{frame}
    \frametitle{Ensemble, tout est possible}
    Nous n'allons pas définir rigoureusement la notion d'ensemble. Un ensemble, c'est un sac, dans lequel on a mis des trucs.

    On note un ensemble entre accolades : $\{\ \}$. Par exemple : $\{3, 3.1, 3.14, 3.141\}$.

    L'ensemble est le bloc de base des mathématiques. 
\end{frame}

\begin{frame}
    \frametitle{Des Ensembles}
    Il y a des ensembles pour à peu près tout : 
    \begin{itemize}
        \item \visible<2->{Des nombres : $\left\{4, 19, 30173116, 19.2459, \pi\right\} $}
        \item \visible<3->{Des chaînes de caractères : $\left\{a, b, "le petit chat est mort", "uuuuuuu"\right\}$}
        \item \visible<4->{Des points du plan : $\left\{(0,0), (42, 0), (31.41, 59.2)\right\}$}
        \item \visible<5->{Des ensembles : $\left\{\{0\}, \{"a"\}, \{19, 8\}\right\}$}
        \item \visible<6->{Un peu de tout à la fois : $\left\{31, "n", \left\{19, "b"\right\}, (51, 17)\right\}$}
    \end{itemize}   
\end{frame}

\begin{frame}
    \frametitle{Plus Généralement}
    Les ensembles écrits comme ci-dessus, par énumération, ce n'est pas très pratique. On va y préférer les ensembles par description (surtout pour les ensembles infinis...)

    \only<2>{Par exemple, on note : $\N$ l'ensemble des nombres entiers et alors $2\N = \left\{2n \mid n \in \N\right\}$ est l'ensemble des nombres pairs.}

    \only<3->{Plus généralement, pour une fonction $f : A \rightarrow B$, on note $f(A) = \left\{f(x) \mid x \in A\right\}$ et $f^{-1}(C) = \left\{x \in A \mid f(x) \in C\right\}$ si $C$ est inclus dans $B$.}

    \only<4>{Donc par exemple : $\left\{x \in \R \mid x^{3} + 2x^{2} - \frac{1}{5}x + 3 = 0\right\}$}
    
\end{frame}

\begin{frame}
    \frametitle{De la Taille des Ensembles}
    \begin{definition}
        Une application $f : A \rightarrow B$ est dite \emph{bijective} si elle permet une correspondance un à un entre $A$ et $B$. Deux ensembles ont même \emph{cardinal} noté $\abs{A}$ s'ils sont en bijection.    
    \end{definition}

    \only<2>{Par exemple, on a une bijection entre $\llbracket 0, 3\rrbracket$ et $\llbracket 1, 4\rrbracket$.}
\end{frame}

\begin{frame}
    \frametitle{De la Taille des Ensembles Infinis}
    \begin{definition}
        \begin{itemize}
            \item Une application $f : A \rightarrow B$ est dite \emph{surjective} si elle prend toute les valeurs dans $B$, donc si $\abs{A} \geq \abs{B}$
            \item \visible<2->{Une application $f : A \rightarrow B$ est dite \emph{injective} si deux éléments de $A$ ont toujours une image distincte, donc si $\abs{A} \leq \abs{B}$}
        \end{itemize}       

    \end{definition}
    
    \visible<3->{Ces définitions sont valables aussi pour des ensembles infinis, et ont un sens dès qu'on dessine des patates.

    Maintenant, on peut partir mesurer l'infini.}
\end{frame}

\subsection{Infinitude}
\begin{frame}
    \frametitle{L'infini de $\N$, intuitif ?}
    $\N$ est infini, et c'est le plus petit ensemble infini, en terme de cardinal. Autrement dit, si $\abs{A} \leq \abs{\N}$, soit il y a égalité, soit $A$ est fini. 
    On dit que $\N$ est infini dénombrable.

    \visible<2->{Que dire de $p : n \in \N \mapsto 2n$ et $i : n \in 2\N \mapsto n + 1 \in 2\N + 1$ ?}
    \visible<3->{Elles sont bijectives. Donc il y a `autant` de nombres pairs que de nombre impairs (logique) que de nombres entiers...}
\end{frame}

\begin{frame}
    \frametitle{L'infini, et au-delà ? $\Z, \N^{2}, \Q$}
    \only<1-2>{On note $\Z$ l'ensemble des nombres relatifs (entiers, négatifs compris).} 
    \only<2-2>{L'application \[
        r : n \in \N \mapsto \begin{cases} n/2 & \text{ si } n \text{ est pair}\\ -\frac{n+1}{2} & \text{ sinon}
        \end{cases}
            \]
        est bijective. Il y a donc autant de nombre relatifs que de nombres positifs.}
    \only<3-4>{On note $\N^{2}$ l'ensemble des couples de nombre entiers, i.e. l'ensemble des points du plan situés à une abscisse entière et une ordonnée entière.}
    \only<4-4>{L'application $\psi : (p, q) \mapsto 2^{p}(2q + 1)$ est bijective. Il y a donc autant de couples d'entiers que d'entiers...}
    
    \only<5>{On note $\Q$ l'ensemble des fractions de nombres entiers, appelé l'ensemble des nombres \emph{rationnels} (qui viennent d'un ration/d'une division). $\Q$ est en bijection avec $\N$, donc est dénombrable.}

    \only<6>{On note $\mathcal{P}$ l'ensemble des nombres premiers. Il est infini et inclus dans $\N$, il est donc en bijection avec $\N$ et est donc dénombrable.}

    \only<7>{Tout est dénombrable alors ? Non : 
    \begin{theorem}
        L'ensemble des parties de $\N$ n'est pas dénombrable, et $\R$ non plus.
    \end{theorem}
    }
\end{frame}

\begin{frame}
    \frametitle{Diagonale}
    \only<1-3>{\begin{proof}
        \only<1>{Supposons qu'on a une bijection de $\N$ dans $\R$. Cela revient à numéroter les nombres réels entre $0$ et $1$, car $\left[0, 1\right]$ et $\R$ sont en bijection.\\}
        \only<2>{On a donc une liste :
        \begin{center}
            \begin{tabular}{lc}
                $0$ & $0.189280493920472\ldots$\\
                $1$ & $0.892019759218405\ldots$\\
                $2$ & $0.249200000000000\ldots$\\
                $3$ & $0.361846238100291\ldots$\\
                $\vdots$ & $\vdots$
            \end{tabular}
        \end{center}
        }
        \only<3>{En prenant le premier chiffre après la virgule du premier nombre, puis le second chiffre du second nombre puis ainsi de suite et en leur ajoutant tous $1$, on crée un nombre qui diffère de chacun des nombres de la liste d'au moins un chiffre. Donc on ne peut pas avoir de bijection... }
    \end{proof}}

    \only<4>{On a utilisé une méthode appelée \emph{Procédé Diagonal de Cantor}. On montre de même que jamais $A$ et $\mathcal{P}(A)$ ne sont en bijection, c'est le théorème de Cantor.\\
    
    On note $\aleph_{0} = \abs{\N} = \abs{\Q} < \abs{\R} = \aleph_{1}$. On appelle ces nombres les premiers et seconds cardinaux infinis et on dit que $\R$ est indénombrable. Il existe des cardinaux infinis encore plus grand. En fait, il en existe une infinité, en considérant $P(P(P(\ldots P(P(\R)))))$ autant de fois qu'on veut.}
    \only<5>{Mézalor, de quoi est composé $\R$ ? \\ 
    
    Si $\Q$ contient les nombres rationnels, $\R \setminus \Q$ est l'ensemble des nombres irrationnels.}
\end{frame}


\section{Le reste de $\R$}
\subsection{Quelques Types de Nombres}
\begin{frame}
    \frametitle{Nombres Irrationnels}
    \begin{definition}
        Un nombre irrationnel est un nombre dont le développement décimal est apériodique et infini (dénombrable !). De manière équivalente, ce sont des nombres qui ne s'écrivent pas sous la forme $\frac{p}{q}$ avec $\text{pgcd}(p, q) = 1$.
    \end{definition}
    
    $\sqrt{2}, e, \pi$ sont irrationnels.
    \only<2>{Rappel : un nombre est pair si et seulement si son carré est pair.}
    \only<3>{\begin{proof}
        Si $\sqrt{2} = \frac{p}{q}$ avec $p \wedge q = 1$. Alors, $2 = \frac{p^{2}}{q^{2}}$ Donc $p^{2}$ est pair et donc $p$ est pair et donc $q$ est impair. Or, avec $p = 2r$, on trouve que $q^{2} = 2r^{2}$ est pair et donc que $q$ est pair.
    \end{proof}}

\end{frame}

\begin{frame}
    \frametitle{(Ir)rationnels}
    \begin{theorem}
        Le développement décimal d'un nombre rationnel est périodique.
    \end{theorem}

    \begin{proof}
        \only<1>{On ne change pas le caractère (a)périodique du développement en multipliant par un nombre entier. On s'intéresse donc uniquement aux développements décimaux de la forme $\frac{1}{q}$.}
        \only<2->{Pour calculer les termes du développement décimal, il suffit de diviser par l'entier $q$ les restes successifs en partant de la première puissance de $10$ supérieure à $q$.}
        \only<3>{\\ Puisqu'il existe un nombre fini de restes (au plus $q-1$), on va nécessairement calculer deux fois un même reste, auquel cas on tombe dans une période du développement.}
    \end{proof}
    

\end{frame}

\begin{frame}
    \frametitle{Nombre de Nombres Irrationnels}
    L'indénombrabilité de $\R$ montre qu'il y a infiniment plus de nombres irrationels que de nombres rationnels. Autrement dit, si on prend un nombre réel `au hasard`, il sera irrationnel.
    \only<1>{En effet, sinon, on pourrait écrire $\R$ comme l'union de deux ensembles dénombrables $\Q$ et $\R\setminus\Q$, et on montre qu'alors $\R$ est dénombrable, en choissant à tour de rôle dans le premier ensemble ou dans le second.}

    \only<2>{Toutefois, on peut toujours approcher un nombre irrationnel par une suite de nombre rationnels : c'est le développement décimal : 
    \begin{theorem}
        Pour $u \in \R$, la suite de terme général $u_{n} = \frac{\lfloor 10^{n}u\rfloor}{10^{n}}$ converge vers $u$.
    \end{theorem}
    On dit que $\Q$ est dense dans $\R$}

    \only<3>{\begin{proof}
        On a : $\abs{u_{n} - u} \leq \frac{1}{10^{n}} \to 0$ par construction.
    \end{proof}}
\end{frame}

\begin{frame}
    \frametitle{Nombres Normaux}
    Les Nombres Normaux, sont des nombres dont les décimales sont équiréparties. C'est à dire, qu'en moyenne, chacun des chiffres de $0$ à $b$ apparaît autant de fois dans leur développement en base $b$: 
    \[
        \lim_{n \to \infty} \frac{b_{n}}{n} = \frac{1}{b}
    \]
    Presque tout nombre réel est un nombre normal.

    Il existe des liens entre ces nombres et la théorie des langages formels, puisque ces nombres sont \og facilement \fg représentés en base dix.    
\end{frame}

\begin{frame}
    \frametitle{Nombres Univers}
    On appelle nombre univers tout nombre qui contient toute suite finie de chiffre.
    Le nombre $123456789101112131415161718192021\ldots$ (constante de Champernowne) est un nombre univers. Depuis l'invention du CD-Rom, on sait que chaque nombre univers contient l'entièreté des livres, films, musiques jamais écrits.
    Presque tout réel est un nombre univers, ce qui fait que si on prend un réel au hasard, il sera un nombre univers.
\end{frame}


\subsection{Quelques Nombres Utiles}
\begin{frame}
    \frametitle{Polynômes}
    \begin{definition}
        Une équation Polynomiale est une équation de la forme : $a_{n}x^{n} + a_{n-1}x^{n-1} + \ldots + a_{1}x + a_{0} = 0$ où les $a_{i}$ sont des nombres appartenant à un ensemble $E$. Elle est dite de degré $n$ sur $E$..
    \end{definition}    

    On va s'intéresser (rapidement) aux solutions de ces équations. 

    \only<2>{Dans le cas $n = 2$, on a : $ax^{2} + bx + c$ et on connaît les solutions : $x_{+, -} = \frac{-b \pm \sqrt{b^{2} - 4ac}}{2a}$}
    \only<3>{
    \begin{theorem}
        Une équation de degré $n$ admet au plus $n$ solutions. Toutes ne sont pas dans $E$.
    \end{theorem}
    }
    
\end{frame}

\begin{frame}
    \frametitle{Nombres Algébriques sur $E$}
    On note $E[X]$ l'ensemble des polynômes sur $E$ (membres de gauche de l'équation). 
    \begin{definition}
        On appelle nombre algébrique sur $E$ tout nombre $x$ tel qu'il existe une équation sur $E$ dont il est solution : 
        \[
            \mathcal{A}(E) = \left\{x \mid \exists \ P \in E[X], P(x) = 0\right\}    
        \]
        Un nombre non algébrique est dit transcendant.
    \end{definition}

    Tous les nombres algébriques ne sont pas dans $E$, mais tout nombre $a$ de $E$ est algébrique car solution de $x - a = 0$.
\end{frame}

\begin{frame}
    \frametitle{Nombres Algébriques}
    On appelle nombre algébrique un nombre algébrique sur $\Q$. 

    \only<1>{$\sqrt{2}, 3, 4/10$ sont algébriques. $e, \pi$ sont transcendants (difficile à prouver). Presque tout réel est transcendant, ce qui fait que si l'on prend un réel au hasard, il sera transcendant.}
    
    \only<2->{On cherche l'ensemble des nombres algébriques. On sait que ce n'est pas $\R$ car il y a des nombres dans $\R$ qui ne sont pas algébriques, et que $x^{2} = -1$ n'a pas de solution sur $\R$.}

    \only<3->{Une équation ayant pour solution $(t_{1}, \ldots, t_{n})$ peut se réduire sous la forme : $(x - t_{1})(x-t_{2})\ldots(x - t_{n}) = 0$. Toutefois si tous les $t_{i}$ ne sont pas dans $\Q$ on ne peut pas autant réduire l'équation.}
\end{frame}

\begin{frame}
    \frametitle{Cloture Algébrique}
    La cloture algébrique $\bar{\Q}$ de $\Q$ est le plus petit ensemble contenant $\Q$ et tel que toute équation sur $\bar{\Q}$ admet au moins une solution.

    \only<1>{Trouver la cloture algébrique d'un ensemble n'est pas un problème très compliqué en théorie.  On peut s'en rendre compte en regardant $\Q[\sqrt{2}] = \left\{a + \sqrt{2}b \mid (a, b) \in \Q^{2}\right\}$ : il suffit de rajouter toutes les solutions aux équations sur $\Q$. Toutefois, en pratique, cela génère des ensembles abstraits et sans forme simple.}
    \only<2>{Un Théorème dû à Kedlaya fait état d'un lien entre la clôture algébrique d'un espace de fonctions, i.e. les solutions à des équations à coefficients fonctionels, et les développements numériques en base $q$. Mais ce théorème qui se base sur la théorie des séries de Laurent et celle des langages formels (voir plus tard...) est hors de notre portée.}
\end{frame}

\section{Extensions}

\subsection{Tous Ensembles Alors ?}

\begin{frame}
    \frametitle{Groupes}
    Un groupe est un ensemble muni d'une loi interne, associative, unifère, inversible. C'est à dire une application $\cdot : G \times G \rightarrow G$ qui vérifie : 
    \begin{itemize}
        \item Pour tous $a, b, c$ dans $G$, $a \cdot (b \cdot c) = (a\cdot b) \cdot c$
        \item Il existe $e \in G$ tel que $e \cdot x = x$ pour tout $x \in G$
        \item Pour tout $x$, il existe $y$ tel que $xy = e$.
    \end{itemize}
    Ici le groupe est noté multiplicativement, mais on aurait pû choisir d'adopter une convention additive, c'est à dire pour une loi $+$.
\end{frame}

\begin{frame}
    \frametitle{Espace Vectoriel}
    Un espace vectoriel réel\footnote{On peut les définir sur tout corps} sur $\R$ est un groupe $E$, noté additivement, stable par multiplication par un réel, i.e. si $x \in E$, $\lambda \in \R$ alors $\lambda x \in E$.
    
    Deux espaces vectoriels sont dit isomorphes s'ils sont en bijection par une application linéaire $f$, i.e. qui préserve la somme.

\end{frame}

\begin{frame}
    \frametitle{Générations}
    Les groupes (resp. les espaces vectoriels) reposent parfois (resp. toujours) sur une famille d'éléments qui les engendre, i.e. un ensemble $\left\{g_{1},\ldots, g_{n}\right\}$ tel que si $g \in G$, $g = g_{1}^{t_{1}}g_{2}^{t_{2}}\ldots g_{n}^{t_{n}}$ pour certains $t_{i}$.
    
    On appelle un tel ensemble une famille génératrice. Une famille génératrice de cardinal minimal est appelée une base.

    \only<2>{On appelle Indice d'un groupe le cardinal d'une base de ce groupe (elles ont toutes le même). C'est une nouvelle notion de taille d'ensemble qui est invariante par la notion d'isomorphisme, une notion de bijection remaniée.}
    
\end{frame}

\begin{frame}
    \frametitle{Dimension}
    Dans le cadre d'un espace vectoriel $E$ sur $\R$, le cardinal d'une base est appelé \emph{dimension} de l'espace et noté $\dim E$. Lorsqu'il est fini, $E$ est de cardinal indénombrable égal à $\aleph_{1}$. Toutefois, on peut montrer que $E$ est en bijection avec $\R^{\dim E}$, et donc, lorsqu'il est infini, selon sa valeur, $E$ va devenir plus ou moins grand.

    \only<2->{Par exemple, si $\dim E = \aleph_{0}$, $E$ est en bijection avec $\R^{\N}$ l'ensemble des suites à valeurs réelles.}

    \only<3->{Lorsque $\dim E = \aleph_{1}$, $E$ est en bijection avec $\R^{\R}$ l'ensemble des fonctions réelles.}

    \only<4->{En réalité, il s'agit plus que d'une bijection, puisqu'il s'agit d'un isomorphisme, mais cela importe peu. }

\end{frame}

\subsection{Un autre nombre}

\begin{frame}
    \frametitle{Nombres de Liouville}
    Un nombre de Liouville est un nombre vérifiant : 
    \[
        \forall n \in \N, \exists\ q_{n} > 1, p_{n}, 0 < \abs{x - \frac{p_{n}}{q_{n}}} <= \frac{1}{q_{n}^{n}}
    \]
    Ce sont des nombres facilement approchés par des nombres rationnels.

    On peut montrer que tout nombre de Liouville est transcendant (voir plus haut), mais que à presque tout nombre réel est transcendant sans être de Liouville.
\end{frame}


\end{document}
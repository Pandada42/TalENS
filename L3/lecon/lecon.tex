\documentclass{scrartcl}
\usepackage{cmap}
\usepackage[T1]{fontenc}
\usepackage[utf8]{inputenc}
\usepackage[french]{babel}
\usepackage{microtype}
\usepackage{enumitem}
\usepackage{hyperref}

\rmfamily
\title{\textbf{\textrm{Exemple de document \LaTeX}}}
\author{Mon nom}
\date{23 Septembre 2022}

\begin{document}
\maketitle

\raggedright\section{\textrm{L'Île mystérieuse}}
\indent{Considérons le texte suivant, extrait de \textit{L'Île Mystérieuse} de Jules Verne. \\}
\begin{quotation}
\og Remontons-nous ?
\begin{itemize}
    \item Non ! Au contraire ! Nous descendons !
    \item Pis que cela, monsieur Cyrus ! Nous tombons !
    \item Pour Dieu ! Jetez du lest !   
    \item Voilà le dernier sac vidé !
    \item Le ballon se relève-t-il ? 
    \item Non !
    \item J’entends comme un clapotement de vagues !
    \item La mer est sous la nacelle !
    \item Elle ne doit pas être à cinq cent pieds de nous !\fg
\end{itemize}

Alors une voix puissante déchira l'air, et ces mots retentirent :\\
\og Dehors tout ce qui pèse ! \ldots ~tout ! et à la grâce de Dieu !\fg
\end{quotation}

\section{\textrm{Intérêts}}
\indent\textbf{Cinéma} Les prix sont les suivants :
\begin{itemize}
    \item 10 € en France
    \item 10 \$ aux Etats-Unis
\end{itemize}

\indent\textbf{Télévision} Rien.

\subparagraph*{Paragraphe sur l'alignement} Ce qui précède est une liste de description, mais ceci est un paragraphe.
\begin{center}
    Voici un texte \textbf{centré}.
\end{center}
\raggedleft Voici un texte aligné à \textsl{droite}\footnote{Ce mot n'est pas en italique, mais en une version \textbf{penchée} de la police --- c'est rarement utilisé.}.

\raggedright
\begin{thebibliography}{2}
    \bibitem[EY09]{EY09} Kousha \textsl{Etessami} et Mihalis \textsl{Yannakakis} : Recursive Markov chains, sto-chastic grammars, and monotone systems of nonlinear equations. \textit{Journal of the ACM}, 56(1):1–66, 2009.
    \bibitem[Knu84]{Knu84} Donald \textsl{E. Knuth} : \textit{The \TeX book}. Addison-Wesley, 1984.
    \bibitem[Pat88]{Pat88} Oren \textsl{Patashnik} : Using BibTeX. Documentation for general Bib\TeX users, janvier 1988.
\end{thebibliography}

\end{document}

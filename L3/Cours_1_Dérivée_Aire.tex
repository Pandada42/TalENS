\documentclass{beamercours}
\usepackage{amsmath}

\title{Cours TalENS 2023-2024}
\subtitle{Dérivée, Volume, Aire, Périmètre}
\date{chaipakan}
\newtheorem{definitionfr}{Définition}


\begin{document}

\maketitle

    \section{Rappels Mathématiques}
        \subsection{Dérivation}
            \begin{frame}{Dérivée par rapport à une variable}
                
                Définition d'une dérivée : Si \mathfont{f} est dérivable, \mathfont{f^{'}(x) = \lim\limits_{\mathrm{d}x \rightarrow 0} \frac{f(x + \mathrm{d}x) - f(x)}{\mathrm{d}x}}
                Toutes les fonctions que nous allons étudier seront dérivables, et même souvent rationnelles.
                Règles de dérivation usuelles : 
                \begin{itemize}
                    \item Si $n$ > 0 : $\deriv{x}x^{n} = nx^{n-1}$
                    \item $\deriv{x} \ln{(x)} = \frac{1}{x}$
                \end{itemize}
            \end{frame}

            
            \begin{frame}{Changement de Variable}   
                
            \end{frame}

        \subsection{Polygones Réguliers et Solides Euclidiens}
            \begin{frame}{Polygones Réguliers : Aire et Périmètre}
                
            \end{frame}
            \begin{frame}{Un catalogue des Solides Euclidiens}
                
            \end{frame}

    \section{Constatations}
        \subsection{En Dimension 2 : Le Cercle}
            \begin{frame}{Rayon, Périmètre, Aire}

            \end{frame}

        \subsection{En Dimension 3 : La Sphère}

        \subsection{Presque Contre-Exemples}
            \begin{frame}{Le Carré}
                
            \end{frame}

            \begin{frame}{Le Triangle Equilatéral}
                
            \end{frame}

            \begin{frame}{Les $n$-gones Réguliers}
                
            \end{frame}

            \begin{frame}{Le Cube}
                
            \end{frame}

    \section{Généralisation}
        \subsection{L'Aire et le Volume en $d$ Dimensions}
            \begin{frame}{Un Espace en $d$ Dimensions ?}
                
            \end{frame}

            \begin{frame}{Un Solide en $d$ Dimensions}
                
            \end{frame}

            \begin{frame}{Aire et Volume d'un Solide en $d$ Dimensions}
                
            \end{frame}

        \subsection{Relation entre Volume et Aire en $d$ Dimensions pour un Solide}
            \begin{frame}{Le cas du Cube}
                
            \end{frame}


        \subsection{Et pour une forme quelconque ?}
            \begin{frame}{Famille Lisse de Formes Uni-Paramétrées}
                
            \end{frame}

            \begin{frame}{Famille Lisse de Formes $k$-Paramétrées}
                
            \end{frame}

\end{document}
